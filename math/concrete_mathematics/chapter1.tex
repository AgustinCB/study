\documentclass{article}

\author{Agustin Chiappe Berrini}
\title{Exercises chapter 1}

\newcounter{problem}

\newcommand\Problem{%
 \stepcounter{problem}%
 \textbf{\theproblem.}~%
}

\newcommand\TheSolution{%
 \textbf{Solution:}\\%
}

\parindent 0in
\parskip 1em
\begin{document}
 \pagenumbering{gobble}
 \maketitle
 \newpage
 \pagenumbering{arabic}

 \Problem All horses are the same color; we can prove this by induction on the number of horses in a given set. Here's how: "If there's just one horsethen it's the same color as itself, so the basis is trivial. For the induction step, assume that there are $n$ horses numbered $1$ to $n$. By the induction hypothesis, horses $1$ through $n-1$ are the same color, and similarly horses $2$ through $n$ are the same color. But the middle horses, $2$ through $n-1$, can't change color when they're in different groups; these are horses, not chameleons. So horses $1$ and $n$ must be the same color as well, by transitivity. Thus all $n$ horses are the same color; QED." What, if anything, is wrong with this reasoning?

 \TheSolution In the use of the condition $p(n-1) \to p(1)$. While it's true that we can say that if $p(n-1)$ then all horses numbered from $1$ to $n-1$ have the same colour, we can't say that horses from $2$ to $n$ are the same colour, as the hourse numbered $n$ is not considered as part of $p(n-1)$, only the first $n-1$ ones.

 \Problem Find the shortest sequence of moves that transfers a tower of $n$ disks from the left peg $A$ to the right peg $B$, if direct moves between $A$ and $B$ are disallowed. (Each move must be to or from the middle peg. As usual, a larger disk must never appear above a smaller one.)

 \TheSolution Let's check first simple cases.

 Assuming $T'(n)$ is the minimum number of steps required under the new rules and that the towers are set in the order $A \to C \to B$, we have that $T'(1) = 2$, given that we need to do the following sequence of movements: $A \to C$, $C \to B$.

 Then we have that $T'(2) = 8$. The steps are "simple:" $A \to C$, $C \to B$ (up to here we move the smallest disk to $B$), $A \to C$ (move the largest disk to the middle tower), $B \to C$, $C \to A$ (move the smallest disk back to $A$), $C \to B$ (move the largest disk to $B$), $A \to C$, $C \to B$ (move the smallest disk to $B$).

 From here, we can infer a solution in five steps:

 \Problem Show that, in the process of transferring a tower under the restrictions of the preceding exercise, we will actually encounter every properly stacked arrangement of $n$ disks on three pegs
\end{document}
